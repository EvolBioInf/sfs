\documentclass[a4paper]{article}
\usepackage{graphics,eurosym,latexsym}
\usepackage{listings}
\lstset{columns=fixed,basicstyle=\ttfamily,numbers=left,numberstyle=\tiny,stepnumber=5,breaklines=true}
\usepackage{times}
\usepackage[round]{natbib}
\bibliographystyle{plainnat}
\oddsidemargin=0cm
\evensidemargin=0cm
\headsep=0cm
\newcommand{\be}{\begin{enumerate}}
\newcommand{\ee}{\end{enumerate}}
\newcommand{\bi}{\begin{itemize}}
\newcommand{\ei}{\end{itemize}}
\newcommand{\I}{\item}
\newcommand{\ty}{\texttt}
\newcommand{\kr}{K_{\rm r}}
\textwidth=16cm
\textheight=23cm
\begin{document}
\title{\ty{sfs} \input{version}: Compute and Bootstrap Site Frequency Spectra}
\author{Bernhard Haubold\\\small Max-Planck-Institute for Evolutionary Biology, Pl\"on, Germany}
\maketitle
\section{Introduction}
For a sample of $n$ haplotypes let $f_i(n)$ be the number of sites where $i$
haplotypes carry a mutation. The vector
\[
f_1(n), f_2(n),...,f_{n-1}(n)
\]
is called the \textit{site frequency spectrum} of the haplotype sample. The package \ty{sfs}
contains two programs, \ty{ms2sfs} and \ty{bootSfs}, for computing and bootstrapping
site frequency spectra. \ty{Ms2sfs}
takes as input multiple haplotype samples simulated with \ty{ms}
\citep{hud02:gen} and prints the corresponding site frequency
spectra. \ty{BootSfs} takes as input one or more site frequency spectra and
bootstraps them.
\section{Getting Started}
\ty{Ms2sfs}  and \ty{bootSfs} were written in C on a computer running Linux and should
work on any standard UNIX system.
However, please contact me at \ty{haubold@evolbio.mpg.de} if you have any problems with the programs.
\bi
\I Download the software
\begin{verbatim}
git clone https://www.github.com/evolbioinf/sfs
\end{verbatim}
\I Change into the downloaded directory
\begin{verbatim}
cd sfs
\end{verbatim}
\I Generate \ty{ms2sfs} \& \ty{bootSfs}
\begin{verbatim}
make
\end{verbatim}
\I Run the test scripts
\begin{verbatim}
make test
\end{verbatim}
The executables \ty{ms2sfs} and \ty{bootSfs} are now located in the
directory \ty{build}. Place them into your PATH.
\item Make the documentation
\begin{verbatim}
make doc
\end{verbatim}
The documentation is now in
\begin{verbatim}
doc/sfs.pdf
\end{verbatim}
\end{itemize}
\subsection{\ty{ms2sfs}}
\begin{itemize}
\I List the options
\begin{verbatim}
ms2sfs -h
\end{verbatim}
\I Test it on a data set consisting of 2 simulated samples of
five (odd) haplotypes.
\begin{verbatim}
ms2sfs data/msOdd.dat
\end{verbatim}
\I Compute the folded site frequency spectrum
\begin{verbatim}
ms2sfs -f data/msOdd.dat
\end{verbatim}
\I Repeat for one sample of six (even) haplotypes
\begin{verbatim}
ms2sfs data/msEven.dat
ms2sfs -f data/msEven.dat
\end{verbatim}
\I Apply \ty{ms2sfs} to $10^4$ simulated haplotypes
\begin{verbatim}
ms 10 10000 -t 10 | ms2sfs | tail
\end{verbatim}
where \ty{ms} is the coalescent simulator by \cite{hud02:gen}.
\ei
\subsection{\ty{bootSfs}}
\begin{itemize}
\item List options
\begin{verbatim}
bootSfs -h
\end{verbatim}
  \I Read one site frequency spectrum and bootstrap it twice:
\begin{verbatim}
bootSfs -i 2 data/test.sfs
\end{verbatim}
\I In case a particular run needs to be duplicated exactly, set the seed
for the random number generator:
\begin{verbatim}
bootSfs -i 2 -s 13 data/test.sfs  
\end{verbatim}
\end{itemize}

\section{Change Log}
\begin{itemize}
\item Version 0.1 (September 25, 2017)
  \begin{itemize}
  \item First working version.
  \end{itemize}
\item Version 0.2 (October 23, 2017)
  \begin{itemize}
  \item Polished interface.
  \end{itemize}
\item Version 0.3 (November 17, 2017)
  \begin{itemize}
  \item Implemented folding of SFS (\ty{-F}).
  \end{itemize}
\item Version 0.4 (November 29, 2017)
  \begin{itemize}
  \item Enable analytic computation of SFS (\ty{-T} to specify
    $\theta$ and \ty{-n} to specify sample size.
  \end{itemize}
\item Version 0.5 (December 1, 2017)
  \begin{itemize}
  \item Fixed error in folding.
  \end{itemize}
\item Version 0.6 (December 18, 2017)
  \begin{itemize}
  \item Cleaned up interface.
  \end{itemize}
\item Version 0.7
  \begin{itemize}
    \item Implemented \ty{-r} for printing raw counts.
  \end{itemize}
\item June 13, 2018
  \begin{itemize}
  \item Posted on \ty{github}.
    \end{itemize}
\end{itemize}
\bibliography{ref}
\end{document}

